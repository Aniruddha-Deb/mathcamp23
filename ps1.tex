\documentclass[12pt]{article}

\usepackage{amsmath, amsthm, amssymb, amsfonts}
\usepackage{thmtools}
\usepackage{graphicx}
\usepackage{setspace}
\usepackage{geometry}
\usepackage{enumitem}
\usepackage{float}
\usepackage{hyperref}
\usepackage[utf8]{inputenc}
\usepackage[english]{babel}
\usepackage{framed}
\usepackage[dvipsnames]{xcolor}
\usepackage{environ}
\usepackage{tcolorbox}

\geometry{
    top=1in,
    bottom=1in,
    right=0.75in,
    left=0.75in,
    headheight=12pt,
}

% ------------------------------------------------------------------------------

\begin{document}
\title{\vspace{-60pt}Winter MathCamp Problemset 1}
\author{Aniruddha Deb}
\date{5 Dec 2023}

\maketitle

\subsection*{Warmup}

\begin{enumerate}[label=Q\arabic*.,resume]
    \item You have 9 coins, one of which is lighter than all the others (call 
        this the dummy coin). You also have a weighing scale. Find the dummy 
        coin in two weighings.
    \item The numbers 1, 2, 3, ... 1984, 1985 are written on a blackboard. We
        decide to erase from the blackboard any two numbers, and replace them
        with their positive difference. After this is done several times, a
        single number remains on the blackboard. Can this number equal 0?
    \item 25 boys and girls each are seated at a round table. Show
        that both neighbors of at least one student are boys.
    \item Can a 5 x 5 square checkerboard be covered by 1 x 2 dominoes?

\end{enumerate}

\subsection*{Some Proofs by Contradiction}

\begin{enumerate}[label=Q\arabic*.,resume]
    \item Prove that the sum of a rational number and an irrational number is 
        irrational 
    \item $\star$ Prove that there are infinitely many prime numbers 
\end{enumerate}

\subsection*{Pigeonhole Principle}

\begin{enumerate}[label=Q\arabic*.,resume]
    \item $\star$ Say you have 6 people at a party, and everyone is either a friend or
        an enemy of each other. Prove that there are either 3 mutual friends,
        or 3 mutual enemies. 

    \item 17 points are distributed inside a square of 1 m. Prove that there
        exist three points which can be covered with a square of side 0.25 m.
\end{enumerate}

\subsection*{Some Proofs by Induction}

\begin{enumerate}[label=Q\arabic*.,resume]
    \item Prove that the nth fibonacci number is given by the formula $$F_n = \frac{\phi^n - \psi^n}{\phi - \psi}$$
        where $\phi = (1+\sqrt{5})/2$ and $\psi = (1-\sqrt{5})/2$. Note that 
        $F_0 = 0$ and $F_1 = 1$.

    \item Prove that 7 divides $4^{n+1} + 5^{2n-1}$ for any positive integer $n$

    \item Prove that every natural number greater than 1 is a product of two 
        or more prime numbers, or is a prime number itself.
        (This is the first part of the fundamental theorem of arithmetic. The 
        second part claims that this factorization is unique. We'll cover that 
        when doing Number Theory)
    
    \item $\star$ Prove that $$\frac{x_1 + x_2 + \ldots + x_n}{n} \ge \sqrt[n]{x_1x_2
        \ldots x_n}$$ for all $n \ge 2$ (This is called the AM-GM inequality. The 
        term on the left is the \emph{arithmetic mean} of $n$ numbers, and 
        the term on the right is the \emph{geometric mean} of $n$ numbers)

\end{enumerate}

\end{document}
